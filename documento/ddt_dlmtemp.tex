\documentclass[12pt]{article}\usepackage[]{graphicx}\usepackage[]{color}
%% maxwidth is the original width if it is less than linewidth
%% otherwise use linewidth (to make sure the graphics do not exceed the margin)
\makeatletter
\def\maxwidth{ %
  \ifdim\Gin@nat@width>\linewidth
    \linewidth
  \else
    \Gin@nat@width
  \fi
}
\makeatother

\definecolor{fgcolor}{rgb}{0.345, 0.345, 0.345}
\newcommand{\hlnum}[1]{\textcolor[rgb]{0.686,0.059,0.569}{#1}}%
\newcommand{\hlstr}[1]{\textcolor[rgb]{0.192,0.494,0.8}{#1}}%
\newcommand{\hlcom}[1]{\textcolor[rgb]{0.678,0.584,0.686}{\textit{#1}}}%
\newcommand{\hlopt}[1]{\textcolor[rgb]{0,0,0}{#1}}%
\newcommand{\hlstd}[1]{\textcolor[rgb]{0.345,0.345,0.345}{#1}}%
\newcommand{\hlkwa}[1]{\textcolor[rgb]{0.161,0.373,0.58}{\textbf{#1}}}%
\newcommand{\hlkwb}[1]{\textcolor[rgb]{0.69,0.353,0.396}{#1}}%
\newcommand{\hlkwc}[1]{\textcolor[rgb]{0.333,0.667,0.333}{#1}}%
\newcommand{\hlkwd}[1]{\textcolor[rgb]{0.737,0.353,0.396}{\textbf{#1}}}%
\let\hlipl\hlkwb

\usepackage{framed}
\makeatletter
\newenvironment{kframe}{%
 \def\at@end@of@kframe{}%
 \ifinner\ifhmode%
  \def\at@end@of@kframe{\end{minipage}}%
  \begin{minipage}{\columnwidth}%
 \fi\fi%
 \def\FrameCommand##1{\hskip\@totalleftmargin \hskip-\fboxsep
 \colorbox{shadecolor}{##1}\hskip-\fboxsep
     % There is no \\@totalrightmargin, so:
     \hskip-\linewidth \hskip-\@totalleftmargin \hskip\columnwidth}%
 \MakeFramed {\advance\hsize-\width
   \@totalleftmargin\z@ \linewidth\hsize
   \@setminipage}}%
 {\par\unskip\endMakeFramed%
 \at@end@of@kframe}
\makeatother

\definecolor{shadecolor}{rgb}{.97, .97, .97}
\definecolor{messagecolor}{rgb}{0, 0, 0}
\definecolor{warningcolor}{rgb}{1, 0, 1}
\definecolor{errorcolor}{rgb}{1, 0, 0}
\newenvironment{knitrout}{}{} % an empty environment to be redefined in TeX

\usepackage{alltt}
\usepackage[latin1]{inputenc} 
\usepackage{amsmath,amsfonts,amssymb,amsthm}
\usepackage[spanish, es-tabla]{babel}
\usepackage[auth-sc,affil-sl]{authblk}
\usepackage[latin1]{inputenc}
\usepackage[spanish]{babel}
\usepackage{graphicx}
\usepackage{natbib}
\usepackage{subfig}

%\usepackage[style=authoryear]{biblatex}
\usepackage[left=2cm,right=2cm,top=2cm,bottom=2cm]{geometry}
\usepackage{setspace}
\usepackage[all]{xy}
%\usepackage{fullpage}
\usepackage{longtable}
%\usepackage {fancyvrb}
\usepackage{fancyhdr}
%\usepackage{color}
\usepackage{pstricks}
\usepackage{amsthm}
%\usepackage{harvard}
\usepackage{float}
\usepackage{footmisc}
%\usepackage[]{draftwatermark}
%\SetWatermarkText{18 de Diciembre-Versi\'on global}
%\SetWatermarkLightness{0.8}
%\SetWatermarkScale{2}

% 
\setlength{\parindent}{0pt}
\setlength{\textwidth}{6.3in}
\setlength{\topmargin}{-33pt}
\setlength{\oddsidemargin}{0pt}
\setlength{\evensidemargin}{0pt}
\setlength{\textheight}{8in}


% cosas del authblk
\setlength{\parskip}{20pt}
\setlength{\affilsep}{1mm}
\renewcommand\Authfont{\small} %\scshape\normalsize
\renewcommand\Affilfont{\itshape\small}

\renewcommand\Authsep{  }
\renewcommand\Authand{ }
\renewcommand\Authands{ }


% definiciones
\newcommand{\keywordname}{Palabras clave}
\newenvironment{keywords}{%
	\paragraph*{\keywordname: }}%
{}{}
\newcommand{\area}{{\bf Area:} {\sc M?todos Matem?tico-Cuantitativos.}}%{\small}

%********************************************%********************************************%********************************************
\IfFileExists{upquote.sty}{\usepackage{upquote}}{}
\begin{document}
	
	\thispagestyle{empty} 
	\begin {center}
	\includegraphics[width=0.40\textwidth]{logo_iesta.png}
		
	UNIVERSIDAD DE LA RE...
	
	Facultad de ....

	Instituto de Estadiística\\
	\vspace{4.5 cm}
\textbf{\Large Titulo}	
\vspace{1.5 cm}
	
\textbf{Manuel};
\textbf{Ignacio}\\
\vspace{0.5 cm}
	
\textbf{Abril, 2019}
\end{center}
\vspace{1.5cm}

\begin{center}
{\Huge \textbf{Serie Documentos de Trabajo}}\\
\vspace{1.0 cm}
\noindent  DT (numero) - ISSN : 1688-6453 
\end{center}
\pagebreak
\thispagestyle{empty} 
\vspace{15.5cm}

Forma de citacin sugerida para este documento: 
\begin{flushleft}
\textbf{queda de postre ... }
\end{flushleft}

\newpage

\setcounter{page}{1} 
\thispagestyle{empty} 

\begin{center}
	\textbf{Titulo ... }
\end{center}

\begin{center}
  Ram?n ?lvarez-Vaz \footnote{\emph{email: }\texttt{ramon@iesta.edu.uy}, ORCID: 0000-0002-2505-4238\label{fn:foot_Ramon}};	Elena Vernazza \footnote{\emph{email: }\texttt{evernazza@iesta.edu.uy}, ORCID: 0000-0003-3123-2165\label{fn:foot_Elena}}\\
		{\small \emph{Instituto de Estad?stica, Facultad de Ciencias Econ?micas y de Administraci?n, Universidad de la Rep?blica}}
\end{center}

\begin{center}
\textbf{RESUMEN}
\end{center}

Contamos con datos de mediciones diarias de temperaturas mínimas y máximas, de una estación meteorológica de Uruguay, desde Enero-1950 a Octubre-2014.

Se define una ola de frío, como un período de tiempo en el cual, la temperatura observada es inferior a un umbral. El objetivo es determinar dicho umbral a través de la estimación del percentil 10 de las temperaturas. Utilizaremos los modelos lineales dinámicos para modelar la serie, y para la estimación de los percentiles.

\section{Olas de frío}

Existen diversas formas de caracterizar una racha de frío, que responden a distintas aplicaciones de su estudio. Las diversas definiciones acuerdan en la necesidad de establecer un umbral de bajas temperaturas (puede ser absoluto o relativo) y en delimitar una ventana de tiempo mínimo. Durante ese tiempo, la temperatura observada debe mantenerse siempre por debajo del umbral definido.

Para el presente trabajo, hemos definido una ola de frío como un período de tiempo mayor o igual a 3 días, en los cuales las temperaturas mínimas y máximas son inferiores a los respectivos percentiles 10 esperados para tales días.

Esta definición requiere de definir para cada día del año un percentil 10 de temperatura mínima y máxima. Para cada $t \in \{1,\dots, 365\}$ definimos el percentil 10 de mínima como $p^n_{10_t}:= \inf\{y: \p(Y^n_t \leq y)\geq 0.1\}$ siendo $Y^n_t$ la temperatura mínima observada en el día $t$. Análogamente definimos el percentil 10 de máxima para el día $t$ como $p^x_{10_t}:= \inf\{y: \p(Y^x_t \leq y)\geq 0.1\}$ donde $Y^x_t$ es la temperatura máxima observada en el día $t$.

Podemos decir entonces que una sucesión de días $t_1, \dots, t_k$ constituyen una ola de frío de largo $k$ si, siendo $k\geq 3$, se cumple simultáneamente que:
$\begin{cases} y^n_{t_i} < p^n_{10_i}  \\ y^x_{t_i} < p^x_{10_i} \end{cases}$ para $i=1,\dots,k$


\textbf{Palabras clave}:  
\textbf{C\'ODIGOS JEL}: 
\vspace{0.5cm}
\textbf{Clasificaci\'on MSC2010}: 

\begin{center}
	\textbf{titulo in english...}
\end{center}

\begin{center}
authors ... 
\end{center}

\begin{center}
	\textbf{ABSTRACT}
\end{center}

oh boy .... 

\textbf{Key words}:  

\textbf{JEL CODES}: 

\textbf{Mathematics Subject Classification MSC2010}: 

\pagebreak
\pagestyle{fancy}
\fancyhf{}
\fancyhead[RE,LO]{titulo }
\fancyhead[LE,RO]{\thepage}
\fancyfoot[RE,RO]{autores abreviados .. }
\fancyfoot[LE,LO]{DT (17/3)-Instituto de Estad?stica}


\renewcommand{\refname}{Referencias Bibliogr\'aficas}

%% Bibliografica usando bibtex
% \pagebreak
% \bibliographystyle{apalike-es}
% \bibliography{biblio/poLCA}

\begin{itemize}
\item Petris, G., Petrone, S., \& Campagnoli, P. (2009). Dynamic linear models. In Dynamic Linear Models with R (pp. 31-84). Springer, New York, NY.

\item Niemi, J. (2012). STAT 615: Advanced Bayesian Methods [Beamer slides]. Retrieved from http://www.jarad.me/courses/stat615/slides/DLMs/DLMs.pdf

\end{itemize}

\pagebreak
\thispagestyle{empty}
\begin{center}

\vspace{1.5 cm}
{\Huge Instituto de Estad?stica} 
\noindent\rule{18cm}{0.4pt}

\vspace{0.5 cm}
\pagestyle{fancy}
{\Huge Documentos de Trabajo}\\
\thispagestyle{empty}
%\noindent\rule{18cm}{0.4pt\includegraphics[width=0.50\textwidth]{grafi/logo_iesta.png}}
\vspace{1.5 cm}
\includegraphics[width=0.40\textwidth]{logo_iesta.png}

%\begin{center}
\thispagestyle{empty}
\vspace{4.5 cm}

\begin{flushright}
	\textbf{\large Eduardo Acevedo 1139. CP 11200
		Montevideo, Uruguay\\
		Tel?fonos y fax: (598) 2410 2564 - 2418 7381\\
		Correo: ddt@iesta.edu.uy\\
		www.iesta.edu.uy\\
		?rea Publicaciones\\}	
\end{flushright}

\vspace{1.5 cm}

\large\textbf{Diciembre, 2017}\\
\large\textbf{DT (17/3)}
\end{center}

\end{document}

